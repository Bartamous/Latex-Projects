\documentclass[12pt, letterpaper]{article}
\usepackage[mathletters]{ucs}
\usepackage[utf8]{inputenc}
\usepackage{csquotes}
\PassOptionsToPackage{hyphens}{url}\usepackage{hyperref}
\hypersetup{colorlinks=true,urlcolor=blue,citecolor=blue,linkcolor=red}
\usepackage{graphicx}

\title{\Huge\center{Parthenon Marbles and Their Repatriation}

\Large\scshape{Ancient History}}
\author{Samim Khaleqi}
\date{\today}
\begin{document}

\maketitle

\pagebreak
\tableofcontents
\pagebreak

\section*{Discuss the conflicting arguments of the stakeholders in the Parthenon Marbles debate.}

\section*{Introduction}
\addcontentsline{toc}{section}{Introduction}
    The Parthenon Marbles are a significant figure in Greek history and their heritage. The Parthenon Marbles, originally from Athens, Greece, were taken by Lord Elgin in 1810, which was under the control of the Ottoman Empire at the time. The marbles are currently residing in the British Museum, where they are subject to repatriation. Repatriation is the return of a thing or person to its country of origin, for which they are being called to return to their native land of the Greeks, but both sides argue they are the owners of the marbles and their property. This essay will argue the legality and complicated moral ethics of the true owners of the Parthenon Marbles.

\section*{Declarations}
\addcontentsline{toc}{section}{Declarations}
    Before continuing with this question, it is important to state that the issues that are obtained from the Parthenon Marbles are not solely associated with emotional or moral issues, though they will also be based on the principles of the law. So any arguments put forward by both parties will be analysed to determine which party has the strongest right to the marbles. Thus, the most important issue of the legitimacy of Lord Elgin’s acquisition and the legitimacy of it will be analysed.

\section*{Argument 1}
\addcontentsline{toc}{section}{Argument 1}
The legal \textbf{repatriation}
\footnote{Reparation: the act of sending or bringing someone, or sometimes money or other property, back to the country that he, she, or it came from.  (Cambridge Dictionary, 2022)}
of the marbles has been sanctioned by the Greek government as an act of theft. Supporters of the Greek Party argue that the marbles were obtained illegally or unethically. It’s stated that the marbles hold significant cultural importance for Greece and that they should be shown among other Parthenon antiquities in the Acropolis Museum. To examine Elgin’s authority to remove the marbles, it is necessary to state if there was any authorisation to take the marbles, the classification of it, and the power it gave Elgin. It is stated that in order to enter the Acropolis site, Lord Elgin acquired a Firman, a Turkish grant, as an outcome of his letter to King Sultan.

\enquote{the friendly request on this point by the said Ambassador, and because it is incumbent that no objection be made to the same [painters] to walk, observe and study the pictures and buildings that they may wish to draw, or to [the implementation and use of] their ladders and instruments, on receipt of the present letter you ensure that, in conformity with the request of the said Ambassador, while the above-mentioned five painters present in the said place are engaged in going in and out of the gate of the Castle of Athens, which is the place of investigation, setting up ladders around the ancient temple of the Idols, moulding with mortar (that is, with plaster) the said ornaments and visible figures, measuring the remains of other ruined buildings, and undertaking when necessary to dig the foundations to find inscribed blocks that may have survived in the gravel, will not be bothered by the Governor of the Castle or by anyone else.

(Signed) Sejid Abdullah Kaimmecam’ 
} 
\footnote{Edhem Eldem, 1999, p.284 . The document was issued in the name of Seyyid Abdullah Pasha kaimakam, identified from Ottoman records as Omer Pa¸sade Elmac Abdullah Pasha, who held the office from 8th ¨ December 1799 until his death on 5th February 1802 from, Edhem El dem (1999). French trade in Istanbul in the eighteenth century. Leiden ; Boston: Brill, p.284. 
}

For which the terms were

\enquote{\begin{itemize}
     \item To enter freely within the walls of the Citadel, and to draw and model with plaster the Ancient Temples there.
    \item To erect scaffolding and to dig where they (Elgin's working team) may wish to discover the ancient foundations.
    \item Liberty to take away any sculptures or inscriptions which do not interfere with the works or walls of the Citadel.
\end{itemize}
} 
\footnote{Smith 1916, p 190}
Though the Firman did not state any action for the removal of the marbles,. The British Government, though, still claims and sustains that the marbles were legally acquired by Lord Elgin, who excavated them from the ruins of the Parthenon, under the discretion of the Sultan, granting permission to remove the marbles. Even though Mr. Abair, the British Ambassador, said in the letter written in 1811 by Lord Elgin, that the Ottoman Government “denied that the persons who had sold those marbles to me (Elgin) had any right to dispose of them ..." (Abair, Letter to Lord Elgin, 1811). This leaves a complex legal debate over documents with moral implications, leaving the marbles in a grey area of balance.

\section*{Argument 2}
\addcontentsline{toc}{section}{Argument 2}
The \textbf{restitution} \footnote{Restitution: the return of objects that were stolen or lost (Cambridge Dictionary, 2022)
}
of the Parthenon Marbles is subjected to a great deal of pressure and argument from both Greek and British parties and stakeholders alike. Cultural objects unlawfully removed from the territory of a Member State \endquote{"shall be returned in accordance with the procedure and in the circumstances provided for in this Directive"}
\footnote{Irini A. Stamatoudi, 1997}
\footnote{Art.2 of Directive 93/7/EEC of 15 March 1993 ((1983) OJ L 74, 27.03.1993, p 74)}. Consequently, if the marbles were taken without permission from Greek territory, it is seen that in the European Directive and International Conventions “The principle of physical return of cultural property is, through increasing State and Institutional practice, becoming a custom of international law" \footnote{Greenfield 1989, p 104}. However, the British government argues that since cultural property does not have an internationally accepted definition, it can be applied according to the point of view of the states in question, therefore, the definition could be widely different for the state that has the property when examined to the state that claims the property. They declare that states can have quite different interpretations of what is cultural property. Britain therefore wonders if the Marbles fit into this category.  Nevertheless, after observing the ministers of culture present at the UNESCO World Conference on Cultural Policies, the Director-General said that the marbles should be restituted back to Greek people(s) to be put back in the structure they were taken from. If this claim is accepted, Britain would need to compensate for the marbles taken, and due to it being unclear if Britain had good intentions for being an innocent purchaser or a valid claim for the acquisition of the marbles, as stated by UNESCO, "the requesting state shall pay just compensation to an innocent purchaser or to a person who has a valid title to that property."\footnote{Art. 77 of the 1970 UNESCO Convention in s. 7(b)(ii)}. This claim, however, has been dismissed by Britain. They contend that maintaining these antiques has required a significant financial and temporal investment. The continuous conflict highlights the precarious equilibrium between practical considerations, legal precepts, and cultural heritage. An article from The Guardian Written by Johnathon Jones, a respected art critic, says that there is a case for the return of the marbles but believes There's a strong enough argument to bring these statues back to Greece, and I don't think the discussion advances either political or cultural causes. He believes that rather than using these pieces of art as cliched conversation starters, people should just quiet up and enjoy them\footnote{(Jones, 2007)}. Therefore, by examining these points, it is shown that the British Museum is in a legal \textbf{quagmire}\footnote{a situation that is hard to deal with or get out of : a situation that is full of problems — usually singular}. Greece claims that the Elgin Marbles is upheld by the ideas of territorial sovereignty and unlawful removal, stating that the return of cultural property not genuine, claiming the \textbf{jus cogas doctrine}\footnote{The Latin expression Jus cogens, also known as Ius cogens, literally translates to "compelling law." It identifies standards that specific agreements cannot deviate from. It comes from the preexisting notion in Roman law that some laws cannot be waived because of the core principles they protect.  Most states and authors agree that jus cogens exists in international law.}, which holds that certain norms are fundamental and cannot be compromised. However, the British Museum is still considering the idea of a legitimate acquisition. Claiming that the marbles were obtained in good faith and with no knowledge of any wrongdoing. Due to the museum's long-term caretaker ship of the marbles, the \textbf{estoppel}\footnote{A rule of evidence or a rule of law that prevents a person from denying the truth of a statement he has made or from denying the existence of facts that he has alleged to exist.} principle is also applicable. Yet, the 1970 UNESCO Convention emphasizes the principle of physical return. In this standoff, the British Museum must choose to justify its legal responsibilities and maintain an overseer responsibility in this problematic legal environment. Finding a clear legal answer to this case while also acknowledging its historical and cultural significance is a significant hurdle that must be overcome to find a solution.

\section*{Argument 3}
\addcontentsline{toc}{section}{Argument 3}
The Parthenon Marbles raise legal questions while also showing a great deal of moral dilemma. Their significance extends beyond national boundaries and has implications for international law. They aren’t just historical artifacts but at the forefront of moral dilemmas, which will be discussed. The British have tried to justify their possession of the marbles, saying he acted in the spirit of a \textbf{preservationist}\footnote{Someone who works to prevent old buildings and areas of the countryside from being destroyed or damaged (Cambridge Dictionary, 2024}, while the Greeks, on the other hand, say the possession of the marbles is unwarranted based on the following ideas. It is said that even if Lord Elgin removed the marbles from the Parthenon statues, he did not intend to gift the marbles to the British or preserve them. The sanctity and safety of the marbles can also be ruled out now since they can be kept at the Acropolis Museum, where they will not be exposed to the environment, and can be displayed appropriately at their site of origin. Ethically, some argue that the marbles are best served by being reunited at the site's origin to preserve cultural heritage and integrity. However, it may set a precedent for museums worldwide due to its emphasis on the importance of rightful ownership and cultural material. Nonetheless, the other party(s) argue and contend that the marbles location in the British Museum allows them to be globally accessed and helps preserve their unilateral significance. Though the Parthenon Marbles haven't been the first case of restitution, numerous antiquities have been given back to their country of origin, such as cultural objects belonging to the Kabaka people of Uganda in 1962, which were returned by the University Museum of Archaeology and Anthropology at Cambridge. Additionally,  in 1942, the London Science Museum returned the Wright Brothers' Kitty-hawk aircraft. If these restitution's have occurred, did they alter the course of history or upend the status quo on a global basis? . "A Museum should not acquire whether by purchase, gift, bequest or exchange, any object unless ... (it is) satisfied that it can acquire a valid title to the object in question and that in particular it has not been acquired in, exported from, its country of origin ... in violation of that country's laws"\footnote{Art.3(2) of the ICOM Code of Professional Ethics of 1986}. Enter Medina Mercouri \textbf{(FIG 1)}, a renowned  actress, singer, activist, and politician. In 1981, she became the first female Minister of Culture and Sports in Greece, for she was a strong advocate for the return of the Parthenon Marbles to Greece, and when her time came in 1982 at the World Conference on Cultural Policies organised by UNESCO in Mexico City, where it was announced that Greece would claim the Parthenon Marbles back, a following response from the UK Arts Minister was that it was a matter of trustees to decide and not elected officials\footnote{Hansard, HC, Vol 38, Col 559–560 (7 March 1983)}. Which caused Medina in May to visit the Parthenon Marbles in the British Museum, on the verge of tears, she can be heard saying, ‘I don’t want to ruin the British Museum’, I want my marbles back. These are part of a unique monument. They have torn down and destroyed this monument.' \footnote{Footage can be seen on the Melina Mercouri Foundation website: melinamercourifoundation.com/en}. After coming to reconciliation, she tried to lodge a more formal claim in 1984 to UNESCO, this time following the proper procedures by the Intergovernmental Committee for Promoting the Return of Cultural Property to its Countries of Origin or its Restitution in Case of Illicit Appropriation (Intergovernmental Committee). A response by the UK stated that again, it was best dealt with by the museum’s Trustees, who were, in any sense, prevented by their governing bodies from returning collection items. Following this the, UK withdrew from UNESCO in 1985, and Melina finished her term in 1989, later dying from cancer in March 1994. With the legitimacy of the marble acquisition shrouded in suspicion, a return of the marbles by the British would not only be in complete compliance with the ICOM's world wide professional ethics, but it would also be seen as good will. 

\section*{Conclusion}
\addcontentsline{toc}{section}{Conclusion}
In this case of the Parthenon Marbles, both the legal and ethical aspects have been analysed, revealing a complex and intricate range of issues. The opposing parties make compelling viewpoints and statements to this claim, yet a true resolution must inevitably favor one side. The arguments of the \textbf{retention}\footnote{The continued possession, use, or control of something.} for Britain's state that retaining the marbles shows Britain's historical ownership, and another claim is the emphasis on the duration of the possession. Unfortunately, both of these arguments fail and lack convincing evidence and robustness when compared to the Greek side. Another argument is based on morals, due to the original reason for retaining the marbles was being justified by their safety and preservation by the British, This reasoning, however, is no longer valid because Greece can now adequately guarantee their well-being. In this complex debate, the fate of the Parthenon Marbles remains contested, intertwining cultural heritage, legal considerations, and ethical dilemmas.

\begin{figure}
    \centering
    \includegraphics{Melina.png}
    \caption{The great campaigner for the return of the Parthenon Marbles, Greek Minister of Culture Melina Mercouri attends the opening of an exhibition on the Acropolis in Amsterdam, 1985}
    \label{Melina}
\end{figure}

\pagebreak
\section*{OPVL Source Analysis}
\addcontentsline{toc}{section}{OPVL Source Analysis}
\subsection*{Origin}
\addcontentsline{toc}{subsection}{Origin}
Author : Johnathon Jones \newline
Date Uploaded : 18 Oct 2007 \newline
Johnathon Jones, has contributed to The Guardian since 1999. He is a significant British art critic, mainly focusing on culture and art. He is also most notably known for being a judge for the Turner Prize in 2009, which is a prestigious award for contemporary artists. Jones is also the author of many books, such as “The Lost Battles: Leonardo, Michelangelo, and the Artistic Duel that Defined the Renaissance” and “The Loves of the Artists: Art and Passion in the Renaissance.”. His critical insights continue to enrich the discourse on art and creativity

\subsection*{Purpose}
\addcontentsline{toc}{subsection}{Purpose}
 Jones wants to give a different perspective on the Parthenon Marbles' return to Greece. He makes the argument that there is not a strong enough reason to bring these antiquated sculptures back to Greece.  He provokes readers' thoughts on nationalism, cultural heritage, and the function of museums by voicing this viewpoint. 

\subsection*{Value}
\addcontentsline{toc}{subsection}{Value}
This source provides insight into the ongoing debate over the Parthenon Marbles, which are currently housed in the British Museum. Jones does acknowledge that there is a case for reuniting the marbles at the Parthenon temple in Athens. Jones’ article contributes to our understanding of the cultural significance of the marbles. He challenges readers to value them more than simple ownership conflicts by highlighting their artistic value.
He asks whether a return is actually necessary, challenging the main notion. By promoting critical thinking regarding cultural heritage and national identity, this deepens the discussion. His argument centers on appreciating the marbles as works of art rather than symbols of national pride.
He questions the continuity of cultural identity between ancient Greece and modern Greece, which adds nuance to the discussion. The reliability of the information presented in the source depends on Jones’ expertise and research. As a respected art critic, his insights carry weight.
Comparing this source with others, we find a range of opinions. Some scholars and activists strongly advocate for repatriation, while others, like Jones, argue for the marbles’ continued display in the British Museum.

\subsection*{Limitations}
\addcontentsline{toc}{subsection}{Limitations}
Jones’ article has its limitations. He critiques the idea of nationalism driving the demand for repatriation. By doing so, he highlights the complexities of cultural identity and historical ownership. Additionally, he emphasizes that Greek classical art belongs to a broader universal legacy, transcending national boundaries and Egocentric perspectives.

\pagebreak
% Glossary
\pagebreak
\section*{\huge\textbf{Glossary}}
\addcontentsline{toc}{section}{Glossary}
\begin{enumerate}
    \item \textbf{Estoppel}: A rule of evidence or a rule of law that prevents a person from denying the truth of a statement he has made or from denying the existence of facts that he has alleged to exist.

    \item \textbf{Repatriation}: The act of sending or bringing someone, or sometimes money or other property, back to the country that he, she, or it came from.

    \item \textbf{Retention}: The continued possession, use, or control of something..
     
    \item \textbf{Restitution}: The return of objects that were stolen or lost.

    \item \textbf{Jus Cogas Doctrine}: The Latin expression Jus Cogens, also known as Ius Cogens, literally translates to "compelling law." It identifies standards that specific agreements cannot deviate from. It comes from the preexisting notion in Roman law that some laws cannot be waived because of the core principles they protect.  Most states and authors agree that jus cogens exists in international law.

    \item \textbf{Preservationist}: Someone who works to prevent old buildings and areas of the countryside from being destroyed or damaged

    \item \textbf{Quagmire}: A situation that is hard to deal with or get out of : a situation that is full of problems — usually singular
\end{enumerate}
\pagebreak

% Referencing
\section*{\huge\textbf{References}}
\addcontentsline{toc}{section}{References}
\begin{enumerate}

    \item Chrysanthi, V. (2022). The Complexities of a Cultural Property Case: An Examination of the Parthenon Marbles. Master's thesis, Harvard University Division of Continuing Education.

    \item Edhem Eldem. (1999). French trade in Istanbul in the eighteenth century (p. 284). Brill.

    \item Greenfield, J (1989) The Return of Cultural Treasures (Cambridge: Cambridge University Press). 
    
    \item Herman, A. (2023). The Parthenon Marbles Dispute. Bloomsbury Publishing.

    \item Jones, J. (2007, October 18). The Parthenon marbles should not be returned to Greece. The Guardian. \url{https://www.theguardian.com/artanddesign/jonathanjonesblog/2007/oct/18/theparthenonmarblesshouldnotbereturnedtogreece} \sloppy
    
    
    \item International Council Of Museums. (1986). The cornerstone of ICOM is the ICOM Code of Ethics for Museums. It sets minimum standards of professional practice and performance for museums and their staff. In joining the organisation, ICOM members undertake to abide by this Code. First published in the three official languages of ICOM, the Code of Ethics has been translated into numerous other languages by ICOM’s committees. iCom Code of ethiCs for museums iCom Code of ethiCs for museums. \url{https://icom.museum/wp-content/uploads/2018/07/ICOM-code-En-web.pdf}

    \item Smith, A.H. (1916). Lord Elgin and his collection. London: [Council of the Society for the Promotion of Hellenic Studies], p.190. 

    \item Stamatoudi, I.A. (1997). Legal and Ethical Issues. [online] 
    \url{ www.parthenon.newmentor.net}
     Availabe at \url{https://www.parthenon.newmentor.net/legal.html}
     
    \item Stefanou, E. (2020). Parthenon (Elgin) Marbles: Case Study. Springer EBooks, 8438–8447. \url{https://doi.org/10.1007/978-3-030-30018-0-2203}

\end{enumerate}
\end{document}
